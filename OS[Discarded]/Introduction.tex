\documentclass[./OperatingSystems.tex]{subfiles}

\begin{document}
\section[Introduction]{Introduction to Operating Systems}

% = = = = = = DEFINITION = = = = = =
\subsection{Definition}
A program that acts as an intemediary between user of a computer and computer hardware.

Goals:

\begin{itemize}
\item Execute user programs
\item Making it easier to use compute hardware
\item Efficient way to use computer hardware
\end{itemize}

% = = = = = = FUNCTIONS = = = = = =
\subsection{Functions}

\begin{mdframed}
A good OS should have following:
\begin{itemize}
\item Ease of use
\item Good Performance
\item Resource Utilization
\end{itemize}    
\end{mdframed}

OS is a resource allocator which manages all resources suchs as CPU time, storage devices etc, and decides between conflicting requests by processes for efficient and fair usage of resource.
It is a control program that controls execution of programs to prevent misuse and managing control over devices.
A \textbf{kernel} is component which keeps running all the time.

\begin{notes}
Bootstrap is loaded at the power-up/reboot and is stored in ROM/EPROM and initializes all aspects of system.
\end{notes}

% = = = = = = STRUCTURE = = = = = =
\subsection{Structure}

\subsubsection*{Multiprogramming\footnote{Batch System}}
One program cannot be keep the CPU busy and thus this calls for some kind of management which in this case is job scheduling.
A subset of all the programs is kept in the memory.One job is selected and run via job scheduling.

\subsubsection*{Time Sharing\footnote{Multitasking}}
This is an extension of multiprogramming in which CPU switches job so frequently that users can interact with each job. Time sharing should be $< 1ms$.
If the processes don't fit in memory, swapping moves them in \& out to run. \textbf{Virtual Memory} allows execution of processes not completely in memory.

% = = = = = = INTERRUPTS = = = = = =
\subsection{Interrupts}

Interrupts alert CPU to events that require attention. Interrupt transfers control to the Interrupt Service Routine generally through 
\textbf{Interrupt vector}, a vector which contains the addresses of all service routines. When a CPU is interrupted, it immidiately transfer execution to that particular location. 
As soon as the interrputed process is completed the CPU should return to the interrupted computation.

\subsubsection*{Interrupt Driven}
\begin{itemize}
\item[-] \textbf{Hardware Interrupt} driven by one of the devices.
\item[-] \textbf{Software Interrupt} (Execution or Trap):
\begin{itemize}
    \item[$\bullet$] Software Error
    \item[$\bullet$] Request for OS Service
    \item[$\bullet$] Other process problems include infinite loop,processes modifying each other or the operating system.
\end{itemize}
\end{itemize}

% = = = = = = OS OPERATIONS = = = = = =
\subsection{OS Operations}

\subsubsection{Dual Mode Operation}


\end{document}

% \definecolor{notecol}{rgb}{0.73,0.56,0.64}
% \newtcolorbox{note}{
%     arc=0pt,
%     boxrule=0pt,
%     colback=notecol,
%     width=\textwidth,   % this option controls the width of the box
%     colupper=white,
%     fontupper=\bfseries
% }

% \begin{center}
% \doublebox{
% \begin{minipage}{\textwidth}
% Over the past twenty years, the study of industrial organization—the analysis of imperfectly competitive markets—has grown from a niche area of microeconomics to a key component of economics.
% \end{minipage}}
% \end{center}