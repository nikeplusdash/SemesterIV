\documentclass[./EngineeringMaths.tex]{subfiles}

\begin{document}
\section[Set Theory]{Set Theory}
A Set is a collection of well defined objects which is denoted by a capital letter and it's elements are described by small letters or numbers.

% = = = = = = TYPES OF SET = = = = = =
\subsection*{Types of Sets}
\begin{itemize}
\item Universal Set ($\xi$ or $U$)
\item Null Set ($\phi$)
\item Subset ($\subset$)
\item Superset ($\supset$)
\item Compliment of a set ($A^c$ or $\bar{A}$)
\item Equal Sets ($=$)
\end{itemize}

% = = = = = = OPERATIONS ON SETS = = = = = =
\subsection*{Operations on Sets}
\begin{itemize}
\item Union ($\cap$)
\item Intersection ($\cup$)
\item De Morgans
\item Laws - Associative, Distributive
\end{itemize}

% = = = = = = TERMINOLOGY = = = = = =
\subsection{Random Experiments, Events and more}
If the repetition of an experiment under identical condition results in different possible outcomes, then such an experiment is called Randome Experiment or Stochastic Experiment.

\textbf{Sample Space (\textbf{S})} is a set of all possible outcomes of a random experiment.\

\textbf{Event ($\mathbf{E}$)} is a subset of Sample Space \textbf{S}

\fbox{Example} Tossing of coin: \textbf{S} = \{H,T\}

\subsection*{Types of Events}

\begin{itemize}
\item Mutually Exclusive
\item Equally Likely
\end{itemize}

\begin{notes}
\textbf{Mutually Exclusive Events}: are events that cannot occur at the same time like tossing of 1 coin can never give both heads and tails.
\vspace{0.5cm}

\textbf{Independent Events}: are events are completely independent of one another like outcome of second toss is independent of the first toss.
\end{notes}
\end{document}