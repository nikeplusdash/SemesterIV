\documentclass[./AlgorithmDesign.tex]{subfiles}

\begin{document}
\section[Introduction]{Introduction to Algorithms}
An algorithm is a sequence of unambiguous instructions for solving a problem i.e. for obtaining a required output
for any legitimate(valid) input in finite time.

% = = = = = = PERFORMANCE/ANALYSIS OF ALGORITHMS = = = = = = 
\subsection{Performance/Analysis of Algorithms}
It refers to the memory and time representation of the program.

Methods of Analysis: 
\begin{itemize}
\item Analytical
\item Experimental
\end{itemize}

Any algorithm is analysed on the following criteria:
\begin{itemize}
\item Space Complexity
\item Time Complexity
\end{itemize}

\subsubsection{Space Complexity}
It is the amount of memory required for a program to completion.
It has 3 categories:

\begin{itemize}
\item Instruction Space (Compiled Program)
\item Data Space (Space needed by var/const)
\item Environment Stack Space (Recursive calls)
\end{itemize}

\fbox{Denoted by: \textbf{$C + S_p$}}

\newpage

\subsubsection*{Sample Questions}
\begin{enumerate}
\item Sum of array without recursion
\begin{mdframed}
\begin{lstlisting}[label=c1]
int sum(int a[],int n) @\label{c1:l1}@
{
    int sum = 0; @\label{c1:l3}@
    for(int i = 0;i < n;i++) @\label{c1:l4}@
        sum = sum + a[i]; 
    return sum; @\label{c1:l6}@
}
\end{lstlisting}

\fbox{Space Complexity: $6x$  bytes\footnote{where $x$ is bytes occupied by int}}
\vspace{3mm}

Reason: 
Line \ref{c1:l1} occupies $x$ bytes for pointer \textit{a} and $x$ bytes for integer \textit{n}. 
Line \ref{c1:l3} occupies $x$ bytes for sum and $x$ bytes for allocating 0. 
Line \ref{c1:l4} will occupy $x$ bytes for allocating integer $i$.
In Line \ref{c1:l6} space will reserved for returning data.
\end{mdframed}

\item Sum of array with recursion
\begin{mdframed}
\begin{lstlisting}[label=c2]
int sum(int a[],int n) @\label{c2:l1}@
{
    if(n > 0)
        return sum(a,n-1) + a[n-1]; @\label{c2:l4}@
    return 0; @\label{c2:l5}@
}
\end{lstlisting}

\fbox{Space Complexity: $3x\times (n + 1)$  bytes\footnote{where $x$ is bytes occupied by int and $n$ is the size of array}}
\vspace{3mm}

Reason: 
Line \ref{c2:l1} occupies $x$ bytes for pointer \textit{a} and $x$ bytes for integer \textit{n}. 
Line \ref{c2:l4} will execute n times and each time space is reserved for pointer $a$ and $n-1$ thus giving $3x\times n$. 
During the last case of $n = 0$, Line \ref{c2:l5} will be executed returing 0 thus occupying $x$ bytes.
\end{mdframed}

\item Linear Search using Recursion 
\begin{mdframed}
\begin{lstlisting}[label=c3]
int search(int a[],int n,int n) 
{
     if(n < 1) return -1;
     if(a[n-1] == x) return x-1;
     search(a,n-1,x); @\label{c3:l5}@
}
\end{lstlisting}

\fbox{Space Complexity: $8x\times (n + 1)$  bytes\footnote{where $x$ is bytes occupied by int and $n$ is the size of array}}
\vspace{3mm}

Reason: 
Line \ref{c3:l5} occupies $8$ bytes for everytime it's executed and similar to previous question we get $n+1$ total executions.
\end{mdframed}
\end{enumerate}

\subsubsection{Time Complexity}

There are 2 approaches to estimate time:
\begin{itemize}
    \item Operation Counter
    \item Step Counter
\end{itemize}

\subsubsection*{Sample Questions}
\begin{enumerate}
\item Finding max element position
\begin{mdframed}
\begin{lstlisting}[label=c4]
int maxpos(int a[],int n) {
    int pos = 0;
    for(int i = 0;i < n;i++)
        if(a[pos] > a[i]) pos = i; @\label{c4:l4}
    return pos;
}
\end{lstlisting}

\fbox{Time Complexity: $n - 1$}
\vspace{3mm}

Reason: 
Line \ref{c4:l4} is executed $n-1$ times.
\end{mdframed}

\item Polynomial Evaluation
\begin{mdframed}
\begin{lstlisting}[label=c5]
int polyeval(int coeff[],int n,int x) {
    int y = 1, value = coeff[0];
    for(int i = 1;i <= n;i++) {
        y = y * x; @\label{c5:l4}@
        value += y * coeff[i]; @\label{c5:l5}@
    }
    return value;
}
\end{lstlisting}

\fbox{Time Complexity: $2n$ Multiplication \& $n$ Addition}
\vspace{3mm}

Reason: 
Line \ref{c5:l4} \& \ref{c5:l5} shows multiplication is done $2n$ times and addition operation is done $n$ times.
\end{mdframed}

\item Polynomial Evaluation using Horner's Algorithm
\begin{mdframed}
\begin{lstlisting}[label=c6]
int horner(int coeff[],int n,int x) {
    int value = coeff[n];
    for(int i = 1;i <= n;i++) 
        value = value * x + coeff[n-i];
    return value;
}
\end{lstlisting}

\fbox{Time Complexity: $n$ Multiplication \& $n$ Addition}
\vspace{3mm}

Reason: 
Let's assume a polynomial $3x^2+3x+1$. By previous method we were simply substituting $x$ into the equation. In case of Horner's Algorithm we simplified the expression i.e. $3x^2+3x+1 = x \underbrace{(3x+3)}_{\textrm{\tiny{evaluated first}}}+1$.
\end{mdframed}

\item Rank Sorting
\begin{mdframed}
\begin{lstlisting}[label=c7]
void rank(int a[],int n,int r[]) {
    for(int i=0;i < n;i++) r[i] = 0;
    for(int i=1;i < n;i++) @\label{c7:l3}@ 
        for(int j = 0;j < i;j++) @\label{c7:l4}@
            if(a[j] <= api])
                r[i]++;
            else
                r[j]++;
}

void rearrange(int a[],int n,int r[]) {
    int *n = new int[n];
    for(int i = 0;i < n;i++) u[r[i]] = a[i]; @\label{c7:l13}@
    for(int i = 0;i < n;i++) a[i] = u[i]; @\label{c7:l14}@
    delete u[];
}
\end{lstlisting}

\fbox{Time Complexity: $\frac{n(n-1)}{2}+2n$}
\vspace{3mm}

Reason: 
As in Line \ref{c7:l3} iterable $i$ goes from 1 to $n-1$ the iterable $j$ in Line \ref{c7:l4} goes from 0 to $i-1$ for every value of $i$ i.e. if $i=1$ then $j: 0\rightarrow0$, if $i=2$ then $j: 0\rightarrow1$ \dots if $i=n-1$ then $j:0\rightarrow n-2$ which can be simplified as sum of number of iterations of j: \[\sum_{i=0}^{n-1} i = \frac{n(n-1)}{2}\]

And from Line \ref{c7:l13} \& \ref{c7:l14} we get the $2n$ operations.
\end{mdframed}

\item  Selection Sort
\begin{mdframed}
\begin{lstlisting}[label=c8]
void SelectionSort(int a[],int n) {
    for(int size = n;size > 1;size--) {
        int j = max(a,size);
        std::swap(a[j],a[size-1]);
    }
}

int max(int a[],int n) {
    int pos = 0;
    for(inr i = 1;i < n;i++)
        if(a[pos] < a[i])
            pos = i;
    return pos;
}
\end{lstlisting}
\fbox{Time Complexity: $\frac{n(n-1)}{2}+3(n-1)$}
\vspace{3mm}

Reason: The $3(n-1)$ factor comes from swapping $n-1$ times while the former follows same pattern as precious Q.\ref{c7}.
\end{mdframed}

\item Transpose with Step-Counter method
\begin{mdframed}
\begin{lstlisting}[label=c9]
void transpose(int **n,int r) {
    for(int i = 0;i < r;i++)
        for(int j = i+1;j < r;j++)
            std::swap(a[i][j],a[j][i]);
}
\end{lstlisting}
\fbox{Time Complexity:  $\frac{(r-1)r}{2} + \frac{r(r+1)}{2} + (r+1)$}
\vspace{3mm}

Reason: 
Let's count the steps, time taken, total execution time with a table.

\begin{center}
\begin{tabular}{|c|c|c|c|}
\hline
Line & $t$ & $\nu$ & $\nu\times t$  \\
\hline
1 & 0 & 0 & 0 \\
2 & 0 & 0 & 0 \\
3 & 1 & $r+1$ & $r+1$ \\
4 & 1 & $\frac{r(r+1)}{2}$ & $\frac{r(r+1)}{2}$\footnote{r+1 because of exit loop condition} \\
5 & 1 & $\frac{(r-1)r}{2}$ & $\frac{(r-1)r}{2}$ \\
\hline
\multicolumn{3}{|c|}{Total Time} & $r^2+r+1$\\
\hline
\end{tabular}
\end{center}
\end{mdframed}
\end{enumerate}

\end{document}

% Template for Sample Q
% \item  
% \begin{mdframed}
% \begin{lstlisting}[label=cX]
    
% \end{lstlisting}
% \fbox{Time Complexity: $\frac{n(n-1)}{2}+2n$}
% \vspace{3mm}

% Reason: 
% \end{mdframed}