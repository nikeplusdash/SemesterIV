\documentclass[./AlgorithmDesign.tex]{subfiles}

\begin{document}
\section[Introduction]{Introduction to Algorithms}
An algorithm is a sequence of unambiguous instructions for solving a problem i.e. for obtaining a required output
for any legitimate(valid) input in finite time.

% = = = = = = PERFORMANCE/ANALYSIS OF ALGORITHMS = = = = = = 
\subsection{Performance/Analysis of Algorithms}
It refers to the memory and time representation of the program.

Methods of Analysis: 
\begin{itemize}
\item Analytical
\item Experimental
\end{itemize}

Any algorithm is analysed on the following criteria:
\begin{itemize}
\item Space Complexity
\item Time Complexity
\end{itemize}

\subsubsection{Space Complexity}
It is the amount of memory required for a program to completion.
It has 3 categories:

\begin{itemize}
\item Instruction Space (Compiled Program)
\item Data Space (Space needed by var/const)
\item Environment Stack Space (Recursive calls)
\end{itemize}

\fbox{Denoted by: \textbf{$C + S_p$}}

\newpage

\subsubsection*{Sample Questions}
\begin{enumerate}
\item Sum of array without recursion
\begin{mdframed}
\begin{lstlisting}[label=c1]
int sum(int a[],int n) @\label{c1:l1}@
{
    int sum = 0; @\label{c1:l3}@
    for(int i = 0;i < n;i++) @\label{c1:l4}@
        sum = sum + a[i]; 
    return sum; @\label{c1:l6}@
}
\end{lstlisting}

\fbox{Space Complexity: $6x$  bytes\footnote{where $x$ is bytes occupied by int}}
\vspace{3mm}

Reason: 
Line \ref{c1:l1} occupies $x$ bytes for pointer \textit{a} and $x$ bytes for integer \textit{n}. 
Line \ref{c1:l3} occupies $x$ bytes for sum and $x$ bytes for allocating 0. 
Line \ref{c1:l4} will occupy $x$ bytes for allocating integer $i$.
In Line \ref{c1:l6} space will reserved for returning data.
\end{mdframed}

\item Sum of array with recursion
\begin{mdframed}
\begin{lstlisting}[label=c2]
int sum(int a[],int n) @\label{c2:l1}@
{
    if(n > 0)
        return sum(a,n-1) + a[n-1]; @\label{c2:l4}@
    return 0; @\label{c2:l5}@
}
\end{lstlisting}

\fbox{Space Complexity: $3x\times (n + 1)$  bytes\footnote{where $x$ is bytes occupied by int and $n$ is the size of array}}
\vspace{3mm}

Reason: 
Line \ref{c2:l1} occupies $x$ bytes for pointer \textit{a} and $x$ bytes for integer \textit{n}. 
Line \ref{c2:l4} will execute n times and each time space is reserved for pointer $a$ and $n-1$ thus giving $3x\times n$. 
During the last case of $n = 0$, Line \ref{c2:l5} will be executed returing 0 thus occupying $x$ bytes.

\end{mdframed}
\end{enumerate}
\end{document}